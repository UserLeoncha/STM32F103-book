\documentclass[UTF8]{ctexart}

\begin{document}

\title{Hello STM32F103}
\author{Leoncha凉茶}
\date{\today}
\maketitle
\centering{生存与发展是第一要务}

\newpage
\begin{abstract}
这是一本为大学生,单片机初学者以及对学习单片机感到无力的人所写的书,如果你不知道该怎么学习,对学习产生了消极情绪,对自己的能力期望过高,请耐心读下去。
  
我们知道,单片机是一块PCB板,有一块或多块主控芯片,一些必备的模块和一些外设,传感器组成。有时这些板子上还带有裸露的接口,这是我们就可以自由的把一些传感器连接到IO口上。

现在我要提出两个问题:什么是单片机?单片机最小系统的三要素是什么?

标准答案是:单片机是一种集成电路芯片。单片机最小系统三要素包括电源电路、晶振电路、复位电路。

这两个问题划清了单片机与普通电路的区别,圈定了单片机的范围,也让单片机在我们的想象中有了形状

对于初学者来说,STM32系列单片机内容繁杂,难以为继,尤其是51都没有搞清楚的小朋友。这是一个致命的问题。首先,要搞清楚STM32由哪些部分构成,都有什么作用,有哪些应用?对于软件、云服务、物联网、机器人,STM32在这些行业中处于怎样的生态位,起到什么作用。

STM32系列单片机内容繁杂,难以为继,尤其是51都没有搞清楚的小朋友。这是一个致命的问题。首先,要搞清楚STM32由哪些模块组成,可以发挥什么作用,可以搭建什么应用?对于软件、云服务、物联网、机器人以及家用电器领域,STM32在单片机行业中处于怎样的生态位?起到什么作用?

知识是信息与相互关系的集合。信息是指经验与总结,是符号;相互关系是指逻辑关系,因果关系。如果想通过前人的总结学会知识,只接收信息是不够的,当你开始询问为什么时,就是在请求逻辑关系与因果关系。

笔者才疏学浅,只能为迷路的初学者指明方向,剩下的一切都要大家自己学习。

下面是模块清单,初学者请跳过:STM32F103芯片、F103板上资源、ARM Cortex-M3架构相关知识(ARM系列处理器的特点,中断触发,寄存器,总线结构)、Keil编译器、STMCubeMX、Linux下开源开发库libopencm3、ST-Link/J-Link、DEBUG、UCOSII、RTOS、SRAM、FLASH、E2PROM、ARM架构采用哈弗架构、改进哈弗架构、冯诺伊曼架构、看门狗(软件开门狗和硬件看门狗)、RTC时钟、SD卡、PWM波(PWM电机、DAC)、LED/LCD、触摸板库、HAL库、CMSIS
STM32内容:GPIO八种模式、时钟树(四个时钟源,两个高速时钟源,两个低速时钟源,一个PLL时钟源,芯片和板上各有一套高低速时钟源)、NVIC中断(四个特点)、Timer(三种定时器)、总线结构(AHB、APB1、APB2,其中APB时钟频率是AHB的二分之一)、通信协议(UART、USART、USMART、I2C、SPI、1-wire、RS232、RS485、USB、2.4GHz、bt、esp8266、zigbee、lora、nb-iot)、流水线
\end{abstract}


\newpage
\section{点灯}
通过配置GPIO来点亮小灯
STM32开发环境搭建、GPIO精讲、原理图
\subsection{内容概述}
\subsubsection{GPIO讲解}
大家在学习、搜集STM32资料时会发现,有的文章中说GPIO有八种工作模式,四种输入、四种输出;有的说是三种输入、两种输出,这些描述无所谓对错,关注这些就走上岔路了。

单片机的硬件基础决定了单片机的工作方式,晶振电路会为芯片提供”心跳“,“心跳”的频率与MCU的处理能力决定了单片机的性能。晶振也决定了单片机处理信号的方式,模拟信号必须转换为数字信号处理,也必须借助数字信号进行输出。否则只能通过设计特殊的电路结构来对信号进行定向处理,但这也将导致该结构不再通用,不过好消息是现在的MCU“心跳”很快,我们暂时无需担心信号来不及采集的问题。

GPIO就是MCU发挥功能的”门户“,其特点鲜明:

1.GPIO只能输出数字信号,对于连续的信号,也只能定时采样,无法真正接受连续的信号。
2.GPIO的结构决定了GPIO的工作模式,输入与输出是将GPIO视为了一根从MCU中伸出的铜线。这根铜线存在两种状态,一种是MCU“监听”这根铜线上的“电平”,另一种是MCU为这根铜线设置一个电平,对外提供电压,我们无需担心电力不足或者电力过高导致的问,这两种状态就是输入与输出。
3.一组GPIO中,各个端口可以分别设置不同的模式。

上述的“输入”与“输出”状态均为概念性描述,接下来解释GPIO八种工作状态:

这里不得不提一句单片机的电源电路(最小系统三要素)了,



\subsubsection{HAL库}
STM32F103并不是空中楼阁,在STM32F103之下,有ARM Cortex-M3架构的支持,CMSIS(ARM Cortex微控制器软件接口标准)就是ARM公司推出维护的,ST公司推出的HAL库就建立在这个标准的基础上,很明显,这些标准之间存在立体的上下相互支持关系。
\subsubsection{delay函数}

\subsection{流程概述}
\subsubsection{初始化}
第一步:初始化hal库
第二步:设置时钟
第三步:初始化delay函数
第四步:初始化LED(自行编写)
\subsubsection{循环部分}
使用HAL库设置两个引脚拉高或拉低,500ms后交换状态
\subsubsection{初始化LED}
1.创建GPIO结构体
2.开启时钟
3.选择GPIO端口,设置GPIO口状态


\newpage
\section{按键输入}
通过配置GPIO捕获按键按下产生的电信号
GPIO,NVIC,按键去抖
\subsection{内容概述}
\subsubsection{NVIC}
ARM Cortex-M3架构 NVIC
\subsection{流程概述}
\subsubsection{初始化}



\subsubsection{时钟树}
时钟树的组成,可以和ARM架构来一起讲,还能再扩展一点,把ARM中断也讲一下。
\subsection{内容讲解}
ARM中断,现场保护,寄存器,时钟树
\subsection{时钟树}
AHB、APB1/APB2、sysclk、systick




\section{流水灯}
使用Timer来控制小灯
\subsection{内容概述}
点灯、NVIC、精讲时钟树、Timer(定时器)
精讲时钟树!
\subsection{流程概述}


\section{PWM电机控制}
使用PWM波控制电机
\subsection{内容概述}
\subsection{流程概述}

\section{看门狗}

\subsection{模块介绍}
IWDG、WWDG、timer
\subsection{内容概述}
\subsubsection{看门狗讲解}
看门狗就是闹钟,看门狗有一个按钮,你可以上设置10s之内必须按一下,否则,闹钟就会响,分为两种,一个是窗口看门狗,一个是独立看门狗

窗口看门狗,之所以称为窗口,是因为其喂狗时间是一个有上下限的范围内,你可以通过设定相关寄存器,设定其上限时间和下限时间:喂狗的时间不能过早也不能过晚。

独立看门狗由内部低速时钟,RC振荡器提供时钟信号。

\subsection{流程概述}




\end{document}
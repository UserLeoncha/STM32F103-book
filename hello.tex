\documentclass[UTF8]{ctexart}


\title{hello STM32F103}

\author{Leon_cha凉茶}
\date{\today}

\begin{document}

\begin{abstract}
对于初学者来说,STM32系列单片机内容繁杂,难以为继,尤其是51都没有搞清楚的小朋友。这是一个致命的问题。首先,要搞清楚STM32由哪些部分构成,都有什么作用,有哪些应用?对于软件、云服务、物联网、机器人,STM32在这些行业中处于怎样的生态位,起到什么作用。

STM32F103系列开发板
STM32F103芯片、F103板上资源、ARM Cortex-M3架构相关知识(ARM系列处理器的特点,中断触发,寄存器,总线结构)、Keil编译器、STMCubeMX、Linux下开源开发库libopencm3、ST-Link/J-Link、DEBUG、UCOSII、RTOS、SRAM、FLASH、E2PROM、ARM架构采用哈弗架构、改进哈弗架构、冯诺伊曼架构、看门狗(软件开门狗和硬件看门狗)、RTC时钟、SD卡、PWM波(PWM电机、DAC)、LED/LCD、触摸板库、HAL库、CMSIS

STM32内容:GPIO八种模式、时钟树(四个时钟源,两个高速时钟源,两个低速时钟源,一个PLL时钟源,芯片和板上各有一套高低速时钟源)、NVIC中断(四个特点)、Timer(三种定时器)、总线结构(AHB、APB1、APB2,APB时钟频率是AHB的二分之一)、通信协议(UART、USART、USMART、I2C、SPI、1-wire、RS232、RS485、USB、2.4GHz、bt、esp8266、zigbee、lora、nb-iot)
\end{abstract}

\section{点灯}
通过阅读原理图,配置GPIO来进行点亮小灯
\subsection{包含模块}
STM32开发环境搭建、GPIO精讲、原理图
\subsubsection{GPIO讲解}
八种模式,四种输入,四种输出,GPIO不能直接输出模拟信号,但是可以对模拟信号进行定时采样,如何保证采样过程中信号不丢失
\subsubsection{点灯流程}
初始化:
第一步:初始化hal库
第二步:设置时钟
第三步:初始化delay函数
第四步:初始化LED
循环部分:
使用HAL库设置两个引脚拉高或拉低,500ms后交换状态
\subsubsection{初始化LED}
1.创建GPIO结构体
2.开启时钟
3.选择GPIO端口,设置GPIO口状态
4.
\subsubsrction{HAL库}
\subsubsection


\section{按键输入}
通过按下按钮来点亮小灯或者触发蜂鸣器,GPIO的高级应用
\subsection{都有什么模块}
NVIC精讲、按钮(键盘、去抖)、GPIO输入模式
\subsection{NVIC}
ARM Cortex-M3架构 NVIC

\section{时钟树}
时钟树的组成,可以和ARM架构来一起讲,还能再扩展一点,把ARM中断也讲一下。
\subsection{模块}
ARM中断,现场保护,寄存器,时钟树
\subsection{时钟树}
AHB、APB1/APB2、sysclk、systick

\section{流水灯}
使用Timer来控制小灯
\subsection{都有什么模块}
点灯、NVIC、精讲时钟树、Timer(定时器)
精讲时钟树!

\section{PWM电机控制}
使用PWM波控制电机
\subsection{模块}


\section{看门狗}
看门狗就是闹钟,看门狗有一个按钮,你可以上设置10s之内必须按一下,否则,闹钟就会响,分为两种,一个是窗口看门狗,一个是独立看门狗

窗口看门狗,之所以称为窗口,是因为其喂狗时间是一个有上下限的范围内,你可以通过设定相关寄存器,设定其上限时间和下限时间:喂狗的时间不能过早也不能过晚。

独立看门狗由内部低速时钟,RC振荡器提供时钟信号。

\subsection{模块介绍}
IWDG、WWDG、timer



\end{document}